\documentclass[11pt,letterpaper]{article}

\usepackage[margin=1in]{geometry}
\usepackage{termcal}
\usepackage{enumitem}
\usepackage{hyperref}
\usepackage{color}

\newcommand{\todo}[1]{\textcolor{red}{TODO: #1}}

\title{AA131: Spaceflight}
\author{Spring 2018}
\date{}

\begin{document}

\maketitle

\section*{Course Description}

This class is all about how to design and build spacecraft. It is designed to introduce undergraduate students to the engineering fundamentals of conceiving, implementing, and operating satellites and other space systems. Topics include orbital and attitude dynamics, mission design, and subsystem technologies. The space environment and the seven classic spacecraft subsystems --- structure, power, propulsion, thermal, attitude determination and control, telemetry and command, and payload --- will be explored. The course will center around a mission design project in which students work in groups to analyze a mission of their choice.

\medskip
\noindent
\textbf{You should take this class if you want to develop a broad understanding of space mission design and spacecraft engineering.} 

\section*{Instructors}

\begin{center}
\begin{tabular}{l c r}
	Prof. Zac Manchester & \textbf{Email:} \href{mailto:zacmanchester@stanford.edu}{zacmanchester@stanford.edu} & \textbf{Office:} Durand 267 \\
	TA: TBD
\end{tabular}
\end{center}

\section*{Logistics}

\begin{itemize}
	\item Lectures will be \todo{TBD} in \todo{TBD}.
	\item Office hours will be held \todo{TBD}.
	\item Homework assignments will be due by \todo{TBD}.
	\item Canvas will be used to distribute course materials and collect assignments.
	\item Slack will be used for general discussion and Q\&A outside of class and office hours.
\end{itemize}

\section*{Course Objectives and Learning Outcomes}
By the end of this course, students should be able to do the following:
\begin{enumerate}
	\item Articulate why we go to space: Identify satellite applications and trade-offs with terrestrial solutions.
	\item Identify the 7 classical spacecraft subsystems, their individual functions, and their interdependencies.
	\item Derive low-level engineering requirements from high-level mission goals.
	\item Analyze subsystem capabilities to produce $\Delta V$, RF link, power, mass, data, and pointing budgets.
\end{enumerate}

\section*{Assignments and Exams}

Each week, as part of the homework, students will be asked to analyze some aspect of their chosen mission. A write-up of this analysis should be completed by each group, and will be reviewed by the instructors and returned to students the following week. At the end of the quarter, these weekly assignments will be compiled into a final report. There will be one midterm exam and no final exam.

\section*{Grading}

Grading will be based on:
\begin{itemize}
	\item 20\% Midterm exam
	\item 30\% Weekly homework assignments
	\item 50\% Completeness, consistency, and quality of the class project
\end{itemize}
Stanford's grading system is defined by the Faculty Senate as A=Excellent, B=Good, C=Satisfactory, D=Minimal Pass, and NP=Not Passed.

\section*{References}

We'll primarily refer to one book during the course: \textit{Spacecraft Mission Engineering: The New SMAD} by Wertz, Everett, and Pushcell. A copy will be placed on reserve in the library, but students are encouraged to purchase their own copy.

\section*{Course Policies}

\textbf{Late Homework:} Students are allowed a budget of 2 late days for turning in homework with no penalty throughout the quarter. They may be used together on one assignment, or separately on two assignments. Beyond these two days, no other late homework will be accepted.

\medskip
\noindent
\textbf{Make-Up Exams:} There will be no make-up exams for the midterm. If extreme circumstances make you unable to attend, let me know as soon as possible. Note the university policy on examinations: ``In submitting official study lists, students commit to all course requirements including the examination procedures chosen and announced by the course instructor.''

\section*{University Policies}

\textbf{The Honor Code:} It is expected that Stanford's Honor Code will be followed in all matters relating to this course. You are encouraged to meet and exchange ideas with your classmates while studying and working on homework assignments, but you are individually responsible for your own work and for understanding the material. You are not permitted to copy or otherwise reference another student's homework or computer code. If you have any questions regarding this policy, feel free to contact the professor.

Compromising your academic integrity may lead to serious consequences, including (but not limited to) one or more of the following: failure of the assignment, failure of the course, disciplinary probation, suspension from the university, or dismissal from the university.

Students are responsible for understanding the University's Honor Code policy and must make proper use of citations of sources for writing papers, creating, presenting, and performing their work, taking examinations, and doing research.

\medskip
\noindent
\textbf{Accommodations:} Students who may need an academic accommodation based on the impact of a disability must initiate the request with the Office of Accessible Education (OAE). Professional staff will evaluate the request with required documentation, recommend reasonable accommodations, and prepare an Accommodation Letter for faculty dated in the current quarter in which the request is being made. Students should contact the OAE as soon as possible since timely notice is needed to coordinate accommodations.


\section*{Schedule of Topics}

\begin{enumerate}[label=\textbf{Week \arabic*:},leftmargin=3.5\parindent]
	\item Course overview: Why we go to space
	\item Orbit mechanics
	\item Orbital maneuvers
	\item Attitude dynamics
	\item Attitude control
	\item Structure and Thermal Systems
	\item Power and Propulsion Systems
	\item Telemetry and Command
	\item The space environment, integration and testing
	\item Spacecraft opperations
\end{enumerate}




\end{document}
