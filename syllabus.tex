\documentclass[11pt,letterpaper]{article}

\usepackage[margin=1in]{geometry}
\usepackage{termcal}
\usepackage{enumitem}
\usepackage{hyperref}
\usepackage{color}
\usepackage{multirow}
\usepackage{multicol}

\newcommand{\todo}[1]{\textcolor{red}{TODO: #1}}

\title{AA131: Space Flight}
\author{Spring 2019}
\date{}

\begin{document}

\maketitle

\section*{Course Description}

Exploring space --- the prospect of traveling beyond the confines of Earth --- inspires engineers like perhaps no other endeavor. This class is all about how (and why) to design and build spacecraft. It is designed to introduce undergraduate students to the engineering fundamentals of conceiving, implementing, and operating satellites and other space systems. Topics include orbital and attitude dynamics, mission design, and subsystem technologies. The space environment and the seven classic spacecraft subsystems --- structure, power, propulsion, thermal, attitude determination and control, telemetry and command, and payload --- will be explored. The course will include both weekly individual homework assignments and a full-quarter mission design project in which students work in groups to analyze a mission of their choice.

\medskip
\noindent
\textbf{You should take this class if you have a passion for spaceflight and want to develop a broad understanding of space mission design and spacecraft engineering.}

\medskip
\noindent
\textbf{Prerequisites:} Freshman-level physics and calculus.

\section*{Instructors}

\begin{center}
\begin{tabular}{l c r}
	Prof. Zac Manchester & \textbf{Email:} \href{mailto:zacmanchester@stanford.edu}{zacmanchester@stanford.edu} & \textbf{Office:} Durand 267 \\
	TA: William Brannon & \textbf{Email:} \href{mailto:wbrannon@stanford.edu}{wbrannon@stanford.edu}
	\\
	TA: Andrew Rapsomanikis & \textbf{Email:} \href{mailto:araps@stanford.edu}{araps@stanford.edu}
\end{tabular}
\end{center}

\section*{Logistics}

\begin{itemize}
	\item Lectures will be held Tuesdays and Thursdays 3:00--4:20 PM in Mitchell B67.
	\item Office hours will be held Tuesdays 1:00--2:30 in Durand 257, Wednesdays 2:00--3:00 in Durand 208, and Thursdays 1:00--2:30 in Durand 208.
	\item Homework assignments will be due by 11:59 PM on Thursdays.
	\item Canvas will be used to distribute course materials and collect assignments.
	\item Slack will be used for general discussion and Q\&A outside of class and office hours.
\end{itemize}

\section*{Course Objectives and Learning Outcomes}
By the end of this course, students should be able to do the following:
\begin{enumerate}
	\item Articulate why we go to space: Identify satellite applications and trade-offs with terrestrial solutions.
	\item Identify the 7 classical spacecraft subsystems, their individual functions, and their interdependencies.
	\item Derive low-level engineering requirements from high-level mission goals.
	\item Analyze and size spacecraft subsystem.
	\item Apply Newton's law of gravitation and Kepler's laws to develop and analyze spacecraft trajectories.
	\item Apply Euler's equation to analyze the attitude dynamics of a spacecraft.
	\item Develop a basic attitude determination system based on vector observations
	
\end{enumerate}

\section*{Assignments and Exams}

Homework will be assigned each week. Each assignment will include individual problems related to the material discussed in class, as well as mission-design questions related to the group project. Collaboration is required for the group project questions, and is encouraged on the individual questions as well. However, each student must hand in their own work.

Homework will be due every week on Thursdays. Solutions and feedback on the group project questions will be provided approximately one week after the due date. At the end of the quarter, the weekly assignments will be compiled into a final report by each group. There will be a midterm exam, a final exam, and a short presentation by each group on their mission-design project.

\section*{Project Guidelines}

Students will form teams of 3-4 to propose a space system of their own design from the list provided at the end of this document. Making this choice will be part of the first homework assignment.

Students will analyze the design of their chosen system in detail throughout the rest of the semester. As the lectures progress, students will return frequently to the analysis behind each subsystem and will correct or adjust it to reflect better understanding of the material.

The final report should consist of roughly 10 pages of content (single spaced, including figures and tables) plus a cover page and references. It should be written as a proposal to a funding agency or company (e.g. NASA, DoD, etc.) and should provide enough detail for a technically competent manager to evaluate the proposed mission. At least 5 technical references must be cited in your report (conference or journal papers, textbooks, etc.). Your group's report must be submitted before your allotted presentation time.

Presentations will be during finals week at a time arranged by each team with the instructors. Each team will have 15 minutes to present roughly 8-12 slides. Projects will be graded based on:
\begin{itemize}
	\item Coverage of the 7 spacecraft subsystems (40\%).
	\item Concept of operations (20\%).
	\item Technical depth and correctness (20\%).
	\item Clarity of Mission Objectives (10\%).
	\item Cost estimates (5\%).
	\item Innovation and creativity (5\%).
\end{itemize}

\section*{Grading}

Grading will be based on:
\begin{itemize}
	\item 30\% Weekly homework assignments
	\item 20\% Midterm exam
	\item 20\% Final exam
	\item 30\% Mission design project
\end{itemize}
Stanford's grading system is defined by the Faculty Senate as A=Excellent, B=Good, C=Satisfactory, D=Minimal Pass, and NP=Not Passed.

\section*{References}

We'll primarily refer to one required text during the course: \textit{Spacecraft Mission Engineering: The New SMAD} by Wertz, Everett, and Pushcell.

\medskip
\noindent
\emph{Stanford University and its instructors are committed to ensuring that all courses are financially accessible to all students. If you are an undergraduate who needs assistance with the cost of course textbooks, supplies, materials and/or fees, you are welcome to approach me directly. If would prefer not to approach me directly, please note that you can ask the Diversity \& First-Gen Office for assistance by completing their questionnaire on course textbooks \& supplies: \href{http://tinyurl.com/jpqbarn}{http://tinyurl.com/jpqbarn} or by contacting Joseph Brown, the Associate Director of the Diversity and First-Gen Office (\href{mailto:jlbrown@stanford.edu}{jlbrown@stanford.edu}; Old Union Room 207). Dr. Brown is available to connect you with resources and support while ensuring your privacy.}

\section*{Course Policies}

\textbf{Late Homework:} Students are allowed a budget of 2 late days for turning in homework with no penalty throughout the quarter. They may be used together on one assignment, or separately on two assignments. Beyond these two days, no other late homework will be accepted.

\medskip
\noindent
\textbf{Make-Up Exams:} There will be no make-up exams for the midterm. If extreme circumstances make you unable to attend, let me know as soon as possible. Note the university policy on examinations: ``In submitting official study lists, students commit to all course requirements including the examination procedures chosen and announced by the course instructor.''

\section*{University Policies}

\textbf{The Honor Code:} It is expected that Stanford's Honor Code will be followed in all matters relating to this course. You are encouraged to meet and exchange ideas with your classmates while studying and working on homework assignments, but you are individually responsible for your own work and for understanding the material. You are not permitted to copy or otherwise reference another student's homework or computer code. If you have any questions regarding this policy, feel free to contact the professor.

Compromising your academic integrity may lead to serious consequences, including (but not limited to) one or more of the following: failure of the assignment, failure of the course, disciplinary probation, suspension from the university, or dismissal from the university.

Students are responsible for understanding the University's Honor Code policy and must make proper use of citations of sources for writing papers, creating, presenting, and performing their work, taking examinations, and doing research.

\medskip
\noindent
\textbf{Accommodations:} Students who may need an academic accommodation based on the impact of a disability must initiate the request with the Office of Accessible Education (OAE). Professional staff will evaluate the request with required documentation, recommend reasonable accommodations, and prepare an Accommodation Letter for faculty dated in the current quarter in which the request is being made. Students should contact the OAE as soon as possible since timely notice is needed to coordinate accommodations.


\section*{Tentative Schedule}

\begin{tabular}{c|c|c|c|c|c}
	Week & Dates & Topics & Reading & HW Out & HW Due \\
	\hline
	\multirow{2}{*}{1} & Apr 2 & Course Overview, Spacecraft Subsystems & 14.1--14.4 & 1 & \\
	 & Apr 4 & The Two Body Problem & JWST & & \\
	\hline
	\multirow{2}{*}{2} & Apr 9 & Orbital Elements & 9.1--9.5 & 2 & 1 \\
	 & Apr 11 & Perturbations & Burns & & \\
	\hline
	\multirow{2}{*}{3}  & Apr 16 & Propulsive Maneuvers & 9.6--9.7 & 3 & 2\\
	 & Apr 18 & Patched Conics & & & \\
	\hline
	\multirow{2}{*}{4}  & Apr 23 & Propulsion Subsystem & 18.1--18.7 & 4 & 3\\
	 & Apr 25 & Attitude Dynamics & & & \\
	\hline
	\multirow{2}{*}{5}  & Apr 30 & Attitude Determination & & & 4\\
	 & May 2 & Attitude Stabilization \& Control & & & \\
	\hline
	\multirow{2}{*}{6}  & May 7 & \textcolor{red}{In-Class Midterm} &  & 5 & \\
	 & May 9 & ADCS Subsystem & 19.1 & & \\
	\hline
	\multirow{2}{*}{7}  & May 14 & Communication &  & 6 & 5\\
	 & May 16 & TT\&C Subsystem & 21.1 & & \\
	\hline
	\multirow{2}{*}{8}  & May 21 & Power Subsystem & 21.2 & 7 & 6\\
	 & May 23 & Structure, Loads, Vibration & 22.1 & & \\
	\hline
	\multirow{2}{*}{9}  & May 28 & Thermal Subsystem & 22.2 & 8 & 7\\
	 & May 30 & The Space Environment & 7.1--7.5 & & \\
	\hline
	10 & Jun 4 & Integration and Test & 23.3--23.4 & & 8\\
\end{tabular}

\section*{Possible Projects}

\textbf{Enceladus Plumes:} The Cassini mission discovered that Saturn's moon Enceladus has geysers of liquid water spraying from its south pole. Design a combination of launch vehicle and spacecraft to analyze the water in the plumes of Enceladus for signs of life.

\medskip
\noindent
\textbf{Venus Atmosphere:} Venus is our closest planetary neighbor, yet there is still much we do not know about its atmosphere. Design a low-cost mission to provide scientists with wind velocity, pressure, and temperature data of the middle atmosphere (10-50 km above the surface) at a high spatial resolution.

\medskip
\noindent
\textbf{Lunar GPS:} Design a constellation of spacecraft to provide positioning and timing information to receivers on or near the surface of the moon. Consider how they might be launched and how they may or may not interact with existing GNSS constellations.

\medskip
\noindent
\textbf{Satellite Servicing:} Design a satellite that can repair either low-Earth orbit or Geostationary satellites. Make sure your design is more cost effective than launching a new satellite.

\medskip
\noindent
\textbf{Wildlife Tracking:} Tracking animal migrations across the globe is a huge challenge for ecologists. Design a small satellite constellation to track the positions of very small low-power radio tags placed on animals such as birds and marine mammals. Consider the cost of deploying the constellation as well as the frequency of position updates.

\medskip
\noindent
\textbf{Earth Video:} A startup called EarthNow recently proposed deploying a satellite constellation that could provide continuous video coverage of the entire Earth. Work through the design of such a constellation. While the cost of this system is not highly constrained, communication bandwidth will likely be a limiting factor.

\medskip
\noindent
\textbf{Laser Sails:} The Breakthrough Starshot project has proposed accelerating gram-scale satellites to 20\% the speed of light using a 100 GW ground-based laser. If such a laser could be built, it could presumably also be used to send much larger spacecraft to destinations within the solar system. Design a spacecraft to cary cargo to Mars using the proposed Starshot laser system.

\medskip
\noindent
\textbf{Ballistic Launch:} Launching satellites without rockets has been a dream of engineers for decades. Recently, a concept known as Quiklaunch has been proposed that uses a light gas gun to launch satellites into low-Earth orbit. Design an Earth observation satellite that can be launched by such a system.


\end{document}
